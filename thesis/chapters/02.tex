\chapter{Analiza tematu}

% \begin{itemize}
%     \item Jaki problem chcę (muszę :-) rozwiązać?
%     \item Dlaczego rozwiązanie problemu jest ważne?
%     \item Jak inni rozwiązują ten problem?
%     \item Jakie są zalety i wady tych rozwiązań?
% \end{itemize}

% Odwołania do literatury:
% książek \cite{bib:ksiazka},
% artykułów w czasopismach \cite{bib:artykul},
% materiałów konferencyjnych \cite{bib:konferencja}
% i stron www \cite{bib:internet}.

% Równania powinny być numerowane
% \begin{align}
%     y = \frac{\partial x}{\partial t}
% \end{align}


% \begin{itemize}
%     \item analiza tematu
%     \item wprowadzenie do dziedziny (\english{state of the art}) – sformułowanie problemu,
%     \item poszerzone studia literaturowe, przegląd literatury tematu (należy wskazać źródła wszystkich informacji zawartych w pracy)
%     \item opis znanych rozwiązań, algorytmów, osadzenie pracy w kontekście
%     \item Tytuł rozdziału jest często zbliżony do tematu pracy.
%     \item Rozdział jest wysycony cytowaniami do literatury \cite{bib:artykul,bib:ksiazka,bib:konferencja}.
%           Cytowanie książki \cite{bib:ksiazka}, artykułu w czasopiśmie \cite{bib:artykul}, artykułu konferencyjnego \cite{bib:konferencja} lub strony internetowej \cite{bib:internet}.
% \end{itemize}

% \begin{Definition}\label{def:1}
%     Definicja to zdanie (lub układ zdań) odpowiadające na pytanie o strukturze „co to jest a?”. Definicja normalna jest zdaniem złożonym z 2 członów: definiowanego (łac. definiendum) i definiującego (łac. definiens), połączonych spójnikiem definicyjnym („jest to”, „to tyle, co” itp.).
% \end{Definition}

% \begin{Theorem}[Pitagorasa]\label{t:pitagoras}
%     W dowolnym trójkącie prostokątnym suma kwadratów długości przyprostokątnych jest równa kwadratowi długości przeciwprostokątnej tego trójkąta.
% \end{Theorem}

% \begin{Example}[generalizacja]\label{ex:generalizacja}
%     Przykładem generalizacji jest para: zwierzę i pies. Pies jest zwierzęciem. Pies jest uszczegółowieniem pojęcia zwierzę. Zwierzę jest uogólnieniem pojęcia pies.
% \end{Example}
