% !TeX spellcheck = pl_PL
%%%%%%%%%%%%%%%%%%%%%%%%%%%%%%%%%%%%%%%%%%%
%                                        %
% Szablon pracy dyplomowej magisterskiej %
% zgodny  z aktualnymi  przepisami  SZJK %
%                                        %
%%%%%%%%%%%%%%%%%%%%%%%%%%%%%%%%%%%%%%%%%%
%                                        %
%  (c) Krzysztof Simiński, 2018-2023     %
%                                        %
%%%%%%%%%%%%%%%%%%%%%%%%%%%%%%%%%%%%%%%%%%
%                                        %
% Najnowsza wersja szablonów jest        %
% podstępna pod adresem                  %
% github.com/ksiminski/polsl-aei-theses  %
%                                        %
%%%%%%%%%%%%%%%%%%%%%%%%%%%%%%%%%%%%%%%%%%
%
%
% Projekt LaTeXowy zapewnia odpowiednie formatowanie pracy,
% zgodnie z wymaganiami Systemu zapewniania jakości kształcenia.
% Proszę nie zmieniać ustawień formatowania (np. fontu,
% marginesów, wytłuszczeń, kursywy itd. ).
%
% Projekt można kompilować na kilka sposobów.
%
% 1. kompilacja pdfLaTeX
%
% pdflatex main
% bibtex   main
% pdflatex main
% pdflatex main
%
%
% 2. kompilacja XeLaTeX
%
% Kompilatacja przy użyciu XeLaTeXa różni się tym, że na stronie
% tytułowej używany jest font Calibri. Wymaga to jego uprzedniego
% zainstalowania.
%
% xelatex main
% bibtex  main
% xelatex main
% xelatex main
%
%
%%%%%%%%%%%%%%%%%%%%%%%%%%%%%%%%%%%%%%%%%%%%%%%%%%%%%
% W przypadku pytań, uwag, proszę pisać na adres:   %
%      krzysztof.siminski(małpa)polsl.pl            %
%%%%%%%%%%%%%%%%%%%%%%%%%%%%%%%%%%%%%%%%%%%%%%%%%%%%%
%
% Chcemy ulepszać szablony LaTeXowe prac dyplomowych.
% Wypełniając ankietę spod poniższego adresu pomogą
% Państwo nam to zrobić. Ankieta jest całkowicie
% anonimowa. Dziękujemy!
%
% https://docs.google.com/forms/d/e/1FAIpQLScyllVxNKzKFHfILDfdbwC-jvT8YL0RSTFs-s27UGw9CKn-fQ/viewform?usp=sf_link
%
%%%%%%%%%%%%%%%%%%%%%%%%%%%%%%%%%%%%%%%%%%%%%%%%%%%%%%%%%%%%%%%%%%%%%%%%%

%%%%%%%%%%%%%%%%%%%%%%%%%%%%%%%%%%%%%%%%%%%%%%%
%                                             %
% PERSONALIZACJA PRACY – DANE PRACY           %
%                                             %
%%%%%%%%%%%%%%%%%%%%%%%%%%%%%%%%%%%%%%%%%%%%%%%

% Proszę wpisać swoje dane w poniższych definicjach.

% dane autora
\newcommand{\FirstNameAuthor}{Rafał}
\newcommand{\SurnameAuthor}{Brauner}
\newcommand{\IdAuthor}{152865}

% drugi autor:
\newcommand{\FirstNameCoauthor}{}
\newcommand{\SurnameCoauthor}{}
\newcommand{\IdCoauthor}{}

\newcommand{\Supervisor}{prof. dr hab. inż.Marcin Grzegorzek}
\newcommand{\Title}{Aplikacja sterująca dronem przy pomocy gestów ręki}
\newcommand{\TitleAlt}{An application that controls the drone using hand gestures}
\newcommand{\Program}{Informatyka}
\newcommand{\Specialisation}{Programowanie systemów inteligentnych}
\newcommand{\Departament}{Katedra Inżynierii Wiedzy}
\newcommand{\Consultant}{}

%%%%%%%%%%%%%%%%%%%%%%%%%%%%%%%%%%%%%%%%%%%%%%%
%                                             %
% KONIEC PERSONALIZACJI PRACY                 %
%                                             %
%%%%%%%%%%%%%%%%%%%%%%%%%%%%%%%%%%%%%%%%%%%%%%%

%%%%%%%%%%%%%%%%%%%%%%%%%%%%%%%%%%%%%%%%

\input{config/settings.tex}
\input{config/my-settings.tex}

%%%%%%%%%%%%%%%%%%%%%%%%%%%%%%%%%%%%%%%%

\begin{document}
%\kslistofremarks

\frontmatter

\input{config/titlepage.tex}

\cleardoublepage
\rmfamily\normalfont
\pagestyle{empty}

\input{chapters/00.tex} % informacje redakcyjne

%%%%%%%%%%%%%%%%%% SPIS TRESCI %%%%%%%%%%%%%%%%%%%%%%
% Add \thispagestyle{empty} to the toc file (main.toc), because \pagestyle{empty} doesn't work if the TOC has multiple pages
\addtocontents{toc}{\protect\thispagestyle{empty}}
\tableofcontents
%%%%%%%%%%%%%%%%%%%%%%%%%%%%%%%%%%%%%%%%%%%%%%%%%%%%%

\setcounter{stronyPozaNumeracja}{\value{page}}
\mainmatter
\pagestyle{empty}
\cleardoublepage
\pagestyle{NumeryStronNazwyRozdzialow}

%%%%%%%%%%%%%% wlasciwa tresc pracy %%%%%%%%%%%%%%%%%

\input{chapters/01.tex}  % wstęp
\input{chapters/02.tex} % [Analiza tematu]
\input{chapters/03.tex} % [Przedmiot pracy]
\input{chapters/04.tex} % Badania
\input{chapters/05.tex} % Podsumowanie

\backmatter

%\bibliographystyle{plplain}  % bibtex
%\bibliography{biblio/biblio} % bibtex
\printbibliography           % biblatex
\addcontentsline{toc}{chapter}{Bibliografia}

\begin{appendices}
    \input{chapters/06.tex} % dokumentacja techniczna
    \input{chapters/07.tex} % Spis skrótów i symboli
    \input{chapters/08.tex} % Lista dodatkowych plików, uzupełniających tekst pracy – jeżeli dotyczny, w przecinym razie – zakomentuj!

    \listoffigures
    \addcontentsline{toc}{chapter}{Spis rysunków}
    \listoftables
    \addcontentsline{toc}{chapter}{Spis tabel}
\end{appendices}

\end{document}
